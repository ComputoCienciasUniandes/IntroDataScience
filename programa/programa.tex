\documentclass[letterpaper,10pt,onecolumn]{article}
\usepackage[spanish]{babel}
\usepackage[utf8]{inputenc}
\usepackage{amsfonts}
\usepackage{amsthm}
\usepackage{amsmath}
\usepackage{mathrsfs}

\usepackage{enumitem}
\usepackage[pdftex]{color,graphicx}
\usepackage{hyperref}
\usepackage{listings}
\usepackage{calligra}
\usepackage{url}
%\usepackage{algpseudocode} 
\DeclareMathAlphabet{\mathcalligra}{T1}{calligra}{m}{n}
\DeclareFontShape{T1}{calligra}{m}{n}{<->s*[2.2]callig15}{}
\newcommand{\scripty}[1]{\ensuremath{\mathcalligra{#1}}}
\lstloadlanguages{[5.2]Mathematica}
\setlength{\oddsidemargin}{0cm}
\setlength{\textwidth}{490pt}
\setlength{\topmargin}{-40pt}
\addtolength{\hoffset}{-0.3cm}
\addtolength{\textheight}{4cm}

\begin{document}
\begin{center}

\includegraphics[width=490pt]{header.png}\\[0.5cm]

\textsc{\LARGE Introducci\'on a la Ciencia de Datos}\\[0.1cm]
%\large Jaime E. Forero Romero\\[0.5cm]

\end{center}

\large \noindent\textsc{Nombre del curso:}  Introducci\'on a la Ciencia de Datos%Aqui  
                                %nombre del curso 
  
\noindent\textsc{Prerrequisito:} FISI 2028 (M\'etodos Computacionales)%Aqui el codigo del

\noindent\textsc{Cr\'editos:} 3 cr\'editos pregrado. 4 cr\'editos posgrado. %Aqui el codigo del

\noindent\textsc{C\'odigo del curso:} FISI 3915 - FISI 4915 %Aqui el codigo del
                                %curso 

\noindent\textsc{Unidad acad\'emica:} Departamento de F\'isica

\noindent\textsc{Periodo acad\'emico:} 202010 %Aqui el periodo,
                                %p.ej. 201510 

\noindent\textsc{Horario:} Mi-Vi, 9:30AM-10:50AM

%\noindent\textsc{Horario (laboratorio - Secci\'on 1):} XXX


\noindent\rule{\textwidth}{1pt}\\[-0.3cm]

\normalsize \noindent\textsc{Nombre profesor magistral:}
Jaime Ernesto Forero Romero%Aqui nombre del profesor principal   

\noindent\textsc{Correo electr\'onico:}
\href{mailto:je.forero@uniandes.edu.co}{\nolinkurl{je.forero@uniandes.edu.co}}
%Cambie address por su direccion de correo uniandes 

\noindent\textsc{Horario y lugar de atenci\'on:}  Lunes 15:00 a 16:00, Oficina Ip208.
%Miercoles 15:00 a 17:00, Oficina Ip208 \\[-0.1cm]

%\normalsize \noindent\textsc{Nombre profesor Laboratorio:}
%Jos\'e Alejandro Monta\~na Cort\'es %Aqui nombre del profesor principal 

%\noindent\textsc{Correo electr\'onico:}
%\href{mailto:ja.montana@uniandes.edu.co}{\nolinkurl{ja.montana@uniandes.edu.co}}
%Cambie address por su direccion de correo uniandes 

%\noindent\textsc{Horario de atenci\'on:} con cita previa. 
%\\[-0.1cm]
%\href{mailto: jd.prada1760@uniandes.edu.co}{\nolinkurl{jd.prada1760@uniandes.edu.co}}

%Cambie address por direccion de correo uniandes del profesor
%complementario 

%\noindent\textsc{Horario y lugar de atenci\'on:} %Aqui horario y
%lugar de atencion del profesor complementario, p.ej. Vi, 15:00 a
%17:00, Oficina Ip102 
%\\[-0.1cm]
%Repetir esto en caso de varios profesores complementarios

%\noindent\textsc{Nombre monitor(a):} %Aqui nombre del monitor si aplica

%\noindent\textsc{Correo electr\'onico:}
%\href{mailto:address@uniandes.edu.co}{\nolinkurl{address@uniandes.edu.co}}
%%Cambie address por direccion de correo uniandes del monitor 

%\noindent\textsc{Horario y lugar de atenci\'on:} %Aqui horario y
%lugar de atencion del monitor, p.ej. Vi, 15:00 a 17:00, Oficina Ip102 

\noindent\rule{\textwidth}{1pt}\\[-0.1cm]

\newcounter{mysection}
\addtocounter{mysection}{1}

\noindent\textbf{\large \Roman{mysection} \quad Introducci\'on}\\[-0.2cm]

%Este espacio es para hacer una introduccion al curso, evidenciando la
%propuesta metodologica. Debe ser clara y precisa. 

\noindent\normalsize 
La ciencia de datos (Data Science) se encuentra hoy en d\'ia en
todas las \'areas t\'ecnicas y cient\'ificas, dentro y afuera del
\'ambito acad\'emico.
Esto se debe en gran parte a que la capacidad de procesar grandes
cantidades de datos en computadoras de alto  rendimiento ha disminuido
en costo monetario y en complejidad. 

El curso de \textit{Introducci\'on a la Ciencia de Datos}
presenta estas posibilidades computacionales a estudiantes de diferentes disciplinas
cient\'ificas. Para esto se propone profundizar sus conocimientos
en dos \'areas: m\'etodos de descripci\'on estad\'istica de datos y la
implementaci\'on de algoritmos para extraer patrones presentes en
diferentes tipos de datos.

Se asume que los estudiantes de este curso ya tienen conocimientos
b\'asicos en m\'etodos computacionales equivalentes al nivel del curso
M\'etodos Computacionales (FISI-2028).
El lenguaje de programaci\'on ser\'a Python.
\\[0.1cm] 

\stepcounter{mysection}
\noindent\textbf{\large \Roman{mysection} \quad Objetivos}\\[-0.2cm]

%En este espacio se debe precisar el ente visor del curso y el
%proposito ideal al finalizar el curso. 
\noindent\normalsize El objetivo principal del curso es presentar
algoritmos y t\'ecnicas b\'asicas para:

\begin{itemize}
\item reducir la complejidad de conjuntos de datos,
\\[-0.6cm]
\item clasificar conjuntos de datos.
\\[-0.6cm]
\end{itemize} 
\vspace*{0.5cm} 

\stepcounter{mysection}
\noindent\textbf{\large \Roman{mysection} \quad Competencias a
  desarrollar}\\[-0.2cm] 

%En este espacio se describen las habilidades que el estudiante desarrollara en el transcurso del curso.

\noindent\normalsize Al finalizar el curso, se espera que el
estudiante est\'e en capacidad de: 

\begin{itemize}
\item describir y clasificar datos con modelos lineales, \\[-0.6cm]   
\item aplicar algoritmos de reducci\'on de dimensionalidad de datos,\\[-0.6cm]  
\item aplicar algoritmos para clasificaci\'on de datos.\\[-0.6cm]
\\[-0.6cm]  
\end{itemize}

\vspace*{0.5cm} 

\stepcounter{mysection}
\noindent\textbf{\large \Roman{mysection} \quad Contenido por
  semanas}\\[-0.2cm]  

%Se expone de forma ordenada toda la tematica a tratar del curso. Debe planearse para 15 semanas.

\noindent\textbf{\textsc{Semana 1.}} 

Presentaci\'on del curso. Ajustes polinomiales. 
\\[-0.3cm] % BISHOP 

\noindent\textbf{\textsc{Semana 2.}} 

Modelos lineales para regresi\'on. 
%Descomposici\'on
%Bias-Varianza. Regresi\'on lineal bayesiana.
%Comparaci\'on de modelos Bayesiana.
\\[-0.3cm] % BISHOP

\noindent\textbf{\textsc{Semana 3}}

Estimaci\'on de par\'ametros Bayesiana.
\\[-0.3cm] % BISHOP

\noindent\textbf{\textsc{Semana 4}}

Ajuste de curvas Bayesiano.
\\[-0.3cm] % BISHOP

\noindent\textbf{\textsc{Semana 5}}

An\'alisis de Componentes Principales.
\\[-0.3cm] % SKIENA 8

\noindent\textbf{\textsc{Semana 6}}

Modelos lineales para clasificaci\'on.
%Funciones discriminantes. Modelos probabil\'isticos generativos.
%Modelos discriminantes probabil\'isticos.
\\[-0.3cm] % BISHOP 3

\noindent\textbf{\textsc{Semana 7}}

Support Vector Machines.
\\[-0.3cm] % BISHOP 4

\noindent\textbf{\textsc{Semana 8}}

\'Arboles de decisi\'on y bosques aleatorios.
\\[-0.3cm] % SKIENA 11.1

\noindent\textbf{\textsc{Semana 9}}

K-means clustering. t-SNE.
\\[-0.3cm]

\noindent\textbf{\textsc{Semana 10}}

(UMAP) Uniform Manifold Approximation and Projection.
%https://towardsdatascience.com/how-exactly-umap-works-13e3040e1668
\\[-0.3cm]


\noindent\textbf{\textsc{Semana 11}}


Perceptron. Redes Neuronales. 
%Feed-forward Networks. Entrenamiento de redes.
%Backpropagation. % BISHOP 5
\\[-0.3cm]

\noindent\textbf{\textsc{Semana 12}}

Pr\'actica sobre Perceptron. Redes Neuronales. 
\\[-0.3cm]


\noindent\textbf{\textsc{Semana 13}}

%Redes Neuronales. Matriz Hessiana. 
Regularizaci\'on en redes neuronales. 
% BISHOP 5
\\[-0.3cm]

\noindent\textbf{\textsc{Semana 14}}

Redes Neuronales Convolucionales. 
% 8.1 Machine Learning Berkeley.
\\[-0.3cm]

\noindent\textbf{\textsc{Semana 15}}

Autoencoders. %14 https://www.deeplearningbook.org/
\\[-0.3cm]

\noindent\textbf{\textsc{Semana 16}}

Generative adversarial networks.
\\[0.1cm]


El repositorio del curso es:
\url{https://github.com/ComputoCienciasUniandes/IntroDataScience}.

\vspace*{0.5cm} 
\stepcounter{mysection}
\noindent\textbf{\large \Roman{mysection} \quad
  Metodolog\'ia}\\[-0.2cm] 

%Se describen las tecnicas y metodos para el desarrollo exitoso del curso.

\noindent\normalsize 

En las sesiones magistrales, luego de presentar un resumen de
los conceptos te\'oricos, se har\'a \'enfasis en la pr\'actica computacional.
Para que esto funcione es necesario que los estudiantes estudien el
tema correspondiente {\bf antes de cada clase} siguiendo las lecturas
preparatorias recomendadas por SICUA.

\vspace*{0.5cm} 
\stepcounter{mysection}
\noindent\textbf{\large \Roman{mysection} \quad Criterios de
  evaluaci\'on}\\[-0.2cm] 

Las componentes que reciben calificaci\'on en la Magistral (en
par\'entesis su contribuci\'on a la nota definitiva) son las
siguientes:  

\begin{itemize}
\item Asistencia ($20\%$). Cada asistencia a clase cuenta como una
  nota de 5.0 y una falta como 0.0. El promedio de esas notas ser\'a
  la nota de asistencia. 
  Si hay {\bf seis} o m\'as fallas no justificadas durante todo el
  semestre esta nota es cero (0.0).
\item Ejercicios ($20\%$ cada uno). En cada clase hay un ejercicio para
  entregar. Cada ejercicio tiene dos partes. 
  La primera se publica al menos un d\'ia antes de la clase y debe
  resolverse por fuera de la magistral. 
  La segunda se publica y se resuelve durante la magistral.
  Durante el semestre el profesor eligir\'a a su discreci\'on cuatro (4)
  de estos ejercicios para ser calificados.
  Solamente se toman en cuenta los ejercicios entregados a trav\'es de
  SICUA en los horarios l\'imites de la actividad.
\end{itemize}

\vspace*{0.5cm} 

\newpage
\stepcounter{mysection}
\noindent\textbf{\large \Roman{mysection} \quad
  Bibliograf\'ia}\\[-0.2cm] 

%Indicar los libros y la documentacion guia.


\noindent\normalsize Bibliograf\'ia principal:

\begin{itemize}


\item
\textit{Pattern Recognition and Machine Learning}. C. M. Bishop, 
Springer, 2006.\\[-0.6cm]

\item 
\textit{The Data Science Manual}. S. S. Skienna, Springer, 2017.\\[-0.6cm]

\item
\textit{Deep Learning}, I. Goodfellow, Y. Bengio A. Courville, MIT Press 2016 \\[-0.6cm]

\item
\textit{A Comprehensive Guide to Machine Learning}, S. Nasiriany, G. Thomas, W. Wang, A. Yang, 
Berkeley, 2019 \\[-0.6cm]

\item
\textit{Python Data Science Handbook.} J. VanderPlas, O'Reilly, 2016.\\[-0.6cm]

\item 
\textit{An Introduction to Statistical Learning with Applications in
  R}, G. James, D. Witten, T. Hastie, R. Tibshirani, Springer, 2015 \\[-0.6cm]

\item 
\textit{A Student's Guide to Numerical  Methods}. I. H. Hutchinson. Cambdrige, 2015 \\[-0.6cm]

\item
\textit{Data Analysis: A Bayesian Tutorial.} D. S. Sivia,
J. Skilling. Second Edition, Oxford Science Publications. 2006 \\[-0.6cm]

\end{itemize}


\end{document}
