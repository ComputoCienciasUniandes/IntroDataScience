\documentclass[letterpaper,10pt,onecolumn]{article}
\usepackage[spanish]{babel}
\usepackage[utf8]{inputenc}
\usepackage{amsfonts}
\usepackage{amsthm}
\usepackage{amsmath}
\usepackage{mathrsfs}

\usepackage{enumitem}
\usepackage[pdftex]{color,graphicx}
\usepackage{hyperref}
\usepackage{listings}
\usepackage{calligra}
\usepackage{url}
%\usepackage{algpseudocode} 
\DeclareMathAlphabet{\mathcalligra}{T1}{calligra}{m}{n}
\DeclareFontShape{T1}{calligra}{m}{n}{<->s*[2.2]callig15}{}
\newcommand{\scripty}[1]{\ensuremath{\mathcalligra{#1}}}
\lstloadlanguages{[5.2]Mathematica}
\setlength{\oddsidemargin}{0cm}
\setlength{\textwidth}{490pt}
\setlength{\topmargin}{-40pt}
\addtolength{\hoffset}{-0.3cm}
\addtolength{\textheight}{4cm}

\begin{document}
\begin{center}

\includegraphics[width=490pt]{header.png}\\[0.5cm]

\textsc{\LARGE Introducci\'on a la Ciencia de Datos}\\[0.1cm]
%\large Jaime E. Forero Romero\\[0.5cm]

\end{center}

\large \noindent\textsc{Nombre del curso:}  Introducci\'on a la Ciencia de Datos%Aqui  
                                %nombre del curso 
  
\noindent\textsc{Prerrequisito:} FISI 2028 (M\'etodos Computacionales) / (M\'etodos computacionales 2)%Aqui el codigo del

\noindent\textsc{Cr\'editos:} 3 cr\'editos pregrado. 4 cr\'editos posgrado. %Aqui el codigo del

\noindent\textsc{C\'odigo del curso:} FISI 3915 - FISI 4915 %Aqui el codigo del
                                %curso 

\noindent\textsc{Unidad acad\'emica:} Departamento de F\'isica

\noindent\textsc{Periodo acad\'emico:} 202210 %Aqui el periodo,
                                %p.ej. 201510 

\noindent\textsc{Horario:} Mi-Vi, 15:30 -16:45

%\noindent\textsc{Horario (laboratorio - Secci\'on 1):} XXX


\noindent\rule{\textwidth}{1pt}\\[-0.3cm]

\normalsize \noindent\textsc{Nombre profesor magistral:}
Jaime Ernesto Forero Romero%Aqui nombre del profesor principal   

\noindent\textsc{Correo electr\'onico:}
\href{mailto:je.forero@uniandes.edu.co}{\nolinkurl{je.forero@uniandes.edu.co}}
%Cambie address por su direccion de correo uniandes 

\noindent\textsc{Horario y lugar de atenci\'on:}  con cita previa.
%Miercoles 15:00 a 17:00, Oficina Ip208 \\[-0.1cm]

\normalsize \noindent\textsc{Nombre profesor Laboratorio:}
John Fredy Su\'arez P\'erez  %Aqui nombre del profesor principal 

\noindent\textsc{Correo electr\'onico:}
\href{mailto:jf.suarez@uniandes.edu.co}{\nolinkurl{jf.suarez@uniandes.edu.co}}
%Cambie address por su direccion de correo uniandes 

\noindent\textsc{Horario de atenci\'on:} con cita previa. 
%\\[-0.1cm]
%\href{mailto: jd.prada1760@uniandes.edu.co}{\nolinkurl{jd.prada1760@uniandes.edu.co}}

%Cambie address por direccion de correo uniandes del profesor
%complementario 

%\noindent\textsc{Horario y lugar de atenci\'on:} %Aqui horario y
%lugar de atencion del profesor complementario, p.ej. Vi, 15:00 a
%17:00, Oficina Ip102 
%\\[-0.1cm]
%Repetir esto en caso de varios profesores complementarios

%\noindent\textsc{Nombre monitor(a):} %Aqui nombre del monitor si aplica

%\noindent\textsc{Correo electr\'onico:}
%\href{mailto:address@uniandes.edu.co}{\nolinkurl{address@uniandes.edu.co}}
%%Cambie address por direccion de correo uniandes del monitor 

%\noindent\textsc{Horario y lugar de atenci\'on:} %Aqui horario y
%lugar de atencion del monitor, p.ej. Vi, 15:00 a 17:00, Oficina Ip102 

\noindent\rule{\textwidth}{1pt}\\[-0.1cm]

\newcounter{mysection}
\addtocounter{mysection}{1}

\noindent\textbf{\large \Roman{mysection} \quad Introducci\'on}\\[-0.2cm]

%Este espacio es para hacer una introduccion al curso, evidenciando la
%propuesta metodologica. Debe ser clara y precisa. 

\noindent\normalsize 
La ciencia de datos (Data Science) se empieza a posicionar en el centro de 
todas las \'areas t\'ecnicas y cient\'ificas, dentro y afuera del
\'ambito acad\'emico.
El curso de \textit{Introducci\'on a la Ciencia de Datos}
presenta un panorama general de los principios y técnicas computacionales 
básicas para una persona que desea iniciarse en la Ciencia de Datos.
Para esto se propone profundizar sus conocimientos
en dos \'areas: estadística descriptiva algoritmos para extraer patrones en
conjuntos de datos.
Se asume que los estudiantes de este curso ya tienen conocimientos
b\'asicos en m\'etodos computacionales equivalentes al nivel del curso
M\'etodos Computacionales (FISI-2028) del antiguo pensum de la carrera de Física, o
Métodos Computacionales 2 del nuevo pénsum.
El lenguaje de programaci\'on ser\'a Python.
\\[0.1cm] 

\stepcounter{mysection}
\noindent\textbf{\large \Roman{mysection} \quad Objetivos}\\[-0.2cm]

%En este espacio se debe precisar el ente visor del curso y el
%proposito ideal al finalizar el curso. 
\noindent\normalsize El objetivo principal del curso es presentar métodos y algoritmos
para extraer conclusiones a partir de un conjunto de datos.
\vspace*{0.5cm} 

\stepcounter{mysection}
\noindent\textbf{\large \Roman{mysection} \quad Competencias a
  desarrollar}\\[-0.2cm] 

%En este espacio se describen las habilidades que el estudiante desarrollara en el transcurso del curso.

\noindent\normalsize Al finalizar el curso, se espera que el
estudiante est\'e en capacidad de: 

\begin{itemize}
\item formular preguntas o hipótesis sobre las propiedades un conjunto de datos. \\[-0.6cm]   
\item  responder preguntas o descartar hipótesis hechas sobre las propiedades un conjunto de datos.\\[-0.6cm]  
\item  comunicar de manera clara las conclusiones de análisis hechos sobre un conjunto de datos.\\[-0.6cm]  
\\[-0.6cm]  
\end{itemize}

\vspace*{0.5cm} 

\stepcounter{mysection}
\noindent\textbf{\large \Roman{mysection} \quad Contenido por
  semanas}\\[-0.2cm]  

%Se expone de forma ordenada toda la tematica a tratar del curso. Debe planearse para 15 semanas.

\noindent\textbf{\textsc{Semana 1.}} 
Presentaci\'on del curso. 
Repaso fundamentos de probabilidad. Repaso de estadística descriptiva. Variables aleatorias. Distribuciones.
\\[-0.3cm] % BISHOP 


\noindent\textbf{\textsc{Semana 2.}}
Prueba de hipótesis. Bootstrapping. 
% BISHOP 5
\\[-0.3cm]

\noindent\textbf{\textsc{Semana 3.}}
Regresión lineal.
\\[-0.3cm]

\noindent\textbf{\textsc{Semana 4.}} 
Presentaciones de los resultados  del proyecto \#1
%Descomposici\'on
%Bias-Varianza. Regresi\'on lineal bayesiana.
%Comparaci\'on de modelos Bayesiana.
\\[-0.3cm] % BISHOP

\noindent\textbf{\textsc{Semana 5.}}
Cadenas de Markov.
\\[-0.3cm] % BISHOP

\noindent\textbf{\textsc{Semana 6.}}
Teorema de Bayes. Estimación de parámetros bayesiana.
\\[-0.3cm] % BISHOP

\noindent\textbf{\textsc{Semana 7.}}
Regresión logística.
\\[-0.3cm] % SKIENA 8

\noindent\textbf{\textsc{Semana 8.}}
Presentaciones de los resultados  del proyecto \#2
%Funciones discriminantes. Modelos probabil\'isticos generativos.
%Modelos discriminantes probabil\'isticos.
\\[-0.3cm] % BISHOP 3

\noindent\textbf{\textsc{Semana 9.}}
Árboles de decisión.
\\[-0.3cm] % BISHOP 4

\noindent\textbf{\textsc{Semana 10}}
Bosques aleatorios.
\\[-0.3cm] % SKIENA 11.1

\noindent\textbf{\textsc{Semana 11.}}
Gradient boosting.
\\[-0.3cm]

\noindent\textbf{\textsc{Semana 12.}}
Presentaciones de los resultados  del proyecto \#3
\\[-0.3cm]

\noindent\textbf{\textsc{Semana 13.}}
Perceptrón
\\[-0.3cm]

\noindent\textbf{\textsc{Semana 14.}}
%Feed-forward Networks. Entrenamiento de redes.
%Backpropagation. % BISHOP 5
Redes neuronales.
\\[-0.3cm]

\noindent\textbf{\textsc{Semana 15.}}
Redes Neuronales Convolucionales.
\\[-0.3cm]

\noindent\textbf{\textsc{Semana 16.}}
Presentaciones de los resultados del proyecto \#4
\\[0.1cm]


El repositorio del curso es:
\url{https://github.com/ComputoCienciasUniandes/IntroDataScience}.

\vspace*{0.5cm} 
\stepcounter{mysection}
\noindent\textbf{\large \Roman{mysection} \quad
  Metodolog\'ia}\\[-0.2cm] 

%Se describen las tecnicas y metodos para el desarrollo exitoso del curso.

\noindent\normalsize 

En las sesiones de los miércoles se hará énfasis en los conceptos teóricos.
En las sesiones de los viernes se hará énfasis en la práctica computacional.
Es necesario que los estudiantes preparen antes de cada clase el
tema correspondiente siguiendo las lecturas
preparatorias recomendadas por SICUA.

\vspace*{0.5cm} 
\stepcounter{mysection}
\noindent\textbf{\large \Roman{mysection} \quad Criterios de
  evaluaci\'on}\\[-0.2cm] 

Se evaluarán 4 Proyectos ($25\%$ cada uno para estudiantes que inscribieron el curso de 3 créditos, $20$\% cada uno para estudiantes que inscribieron el curso de 4 créditos). Cada cuatro semanas los estudiante deben presentar la solución a una pregunta con las herramientas vistas  hasta el momento en el curso. 
Se evalúa la originalidad de la pregunta, la claridad del planteamiento de la pregunta, la relevancia del pregunta, la solidez de los experimentos computacionales para responderla y la claridad de la explicación de los resultados.

Para los estudiantes que inscribieron el curso de 4 créditos hay un $20\%$ que se califica a partir de la toma de apuntes. Cada semana habrá tres estudiantes encargados de tomar los apuntes de la clase para pasarlos a \LaTeX en un overleaf del curso.

\vspace*{0.5cm} 


\stepcounter{mysection}
\noindent\textbf{\large \Roman{mysection} \quad
  Bibliograf\'ia}\\[-0.2cm] 

%Indicar los libros y la documentacion guia.


\noindent\normalsize Bibliograf\'ia principal:

\begin{itemize}

\item 
\textit{Introduction to Probability and Statistics for Engineers and
  Scientists}. Sheldon M. Ross. Third Edition. Elsevier Academic
  Press. 2004.\\[-0.6cm]

\item
\textit{Pattern Recognition and Machine Learning}. C. M. Bishop, 
Springer, 2006.\\[-0.6cm]

\item 
\textit{The Data Science Manual}. S. S. Skienna, Springer, 2017.\\[-0.6cm]

\item
\textit{Deep Learning}, I. Goodfellow, Y. Bengio A. Courville, MIT Press 2016 \\[-0.6cm]

\item
\textit{A Comprehensive Guide to Machine Learning}, S. Nasiriany, G. Thomas, W. Wang, A. Yang, 
Berkeley, 2019 \\[-0.6cm]

\item
\textit{Python Data Science Handbook.} J. VanderPlas, O'Reilly, 2016.\\[-0.6cm]

\item 
\textit{An Introduction to Statistical Learning with Applications in
  R}, G. James, D. Witten, T. Hastie, R. Tibshirani, Springer, 2015 \\[-0.6cm]

\item 
\textit{A Student's Guide to Numerical  Methods}. I. H. Hutchinson. Cambdrige, 2015 \\[-0.6cm]

\item
\textit{Data Analysis: A Bayesian Tutorial.} D. S. Sivia,
J. Skilling. Second Edition, Oxford Science Publications. 2006 \\[-0.6cm]

\end{itemize}


\end{document}
